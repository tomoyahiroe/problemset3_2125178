% !TEX program = lualatex
\documentclass{article}
\usepackage{fontspec}
\usepackage{luatexja-fontspec}
\usepackage{luatexja-ruby} % ルビを使用する場合に必要
\usepackage{graphicx}
\usepackage{listings}
\usepackage{amsmath}
\usepackage[top=25truemm,bottom=20truemm,left=20truemm,right=20truemm]{geometry}
%\usepackage{xcolor}

% フォントの設定
%\setmainfont{Noto Serif JP}    % 通常の明朝体
%\setsansfont{Noto Sans JP}     % 通常のゴシック体
%\setmonofont{Noto Sans Mono JP} % 等幅フォント
%\setmainjfont{Noto Serif JP}   % 日本語の明朝体
%\setsansjfont{Noto Sans JP}    % 日本語のゴシック体
%\setmonojfont{Noto Sans Mono JP} % 日本語の等幅フォント

% 図の参照パス
% \graphicspath{{../01_code/output/}}

% コードのスタイル設定
\lstset{
  basicstyle=\ttfamily\small, % フォントのサイズとスタイル
  %keywordstyle=\color{blue},  % キーワードの色
  %commentstyle=\color{gray},  % コメントの色
  %stringstyle=\color{red},    % 文字列の色
  numbers=left,               % 行番号を左側に表示
  %numberstyle=\tiny\color{gray}, % 行番号のスタイル
  breaklines=true,            % 長い行を折り返す
	frame=single,               % コードブロックに枠をつける
}
\renewcommand\thesubsection{\Alph{subsection}}
\begin{document}
\parindent = 0pt

% タイトル
\title{ミクロデータサイエンス\\Problemset3}
\author{2125178\\廣江友哉}
\date{\today}
\maketitle


% 段落
\section{Step 2 回帰分析}

\subsection{write\_regression\_models の修正}

\begin{lstlisting}
  write_regression_models <- function() {
    regression_models <- list(
      # correct model
      "(1)" = log10(income_child) ~ effort + log10(income_parent),

      # ommited variables models
      # model 2
      "(2)" = log10(income_child) ~ log10(income_parent),
      # model 3
      "(3)" = log10(income_child) ~ effort,

      # measurementerror models
      # model 4
      "(4)" = log10(income_child_noisy) ~ effort + log10(income_parent),
      # model 5
      "(5)" = log10(income_child) ~ effort_noisy + log10(income_parent),
      # model 6
      "(6)" = log10(income_child) ~ effort + log10(income_parent_noisy)
    )

    return(regression_models)
  }
\end{lstlisting}

\subsection{回帰分析表のアウトプット}

\includegraphics[width=18cm]{regression_table.png}

\subsection{モデル1の推定値の95\%信頼区間}

\begin{centering}
  \begin{gather}
    1.09 - 1.96 \times SE(\hat{\beta_1}) \le \hat{\beta_1} \le 1.09 + 1.96 \times SE(\hat{\beta_1}) \\
    1.09 - 1.96 \times 0.17 \le \hat{\beta_1} \le 1.09 + 1.96 \times 0.17 \\
    0.7568 \le \hat{\beta_1} \le 1.4232
  \end{gather}
\end{centering}

\begin{centering}
  \begin{gather}
    0.39 - 1.96 \times SE(\hat{\beta_2}) \le \hat{\beta_2} \le 0.39 + 1.96 \times SE(\hat{\beta_2}) \\
    0.39 - 1.96 \times 0.15 \le \hat{\beta_2} \le 1.09 + 1.96 \times 0.15 \\
    0.096 \le \hat{\beta_2} \le 0.684
  \end{gather}
\end{centering}

\subsection{帰無仮説 $\beta_1 = 0$ と $\beta_2 = 0$ をそれぞれ棄却できるか }

有意水準を5\%で検定をすると,$\beta_1 = 0$ は 95\%の信頼区間から外れるため,帰無仮説は棄却される.\\
同様にして,$\beta_2 = 0$は95\%の信頼区間から外れるため,帰無仮説は棄却される.

\subsection{帰無仮説 $\beta_1 = 1$ を棄却できるか }

\subsection{モデル2,モデル5の解釈}

モデル2は “努力” が欠落した分,親の所得のlog の係数 $\hat{\gamma_2}$ は 0.63 と 真のモデルの係数 $\hat{\beta_2}$ の 0.39 よりも大きい値となっている.つまり,このモデルには欠落変数バイアスが存在する.

モデル5は “努力” に測定誤差があることで,真のモデルと比較して係数が 0.47 と低い値となている.これは測定誤差が説明変数にある時に発生する減衰バイアスである.



\end{document}



